% Please add the following required packages to your document preamble:
% \usepackage[table,xcdraw]{xcolor}
% If you use beamer only pass "xcolor=table" option, i.e. \documentclass[xcolor=table]{beamer}
\newcommand{\disregardCell}{\cellcolor[HTML]{C0C0C0}}
\begin{table}[!htbp]
\centering
\caption{Design criteria}
\label{tab:desigCrit}
\begin{tabular}{|p{3cm} |p{13.8cm}|}
\hline
\cellcolor[HTML]{\CellColor}\textbf{Design criteria}   & \cellcolor[HTML]{\CellColor}\textbf{Reasoning}
\\ \hline
\textbf{Performance}: reach ground speed higher than wind speed    
& 
In this project, the term `performance' relates directly to how fast the vehicle can travel downwind, compared to the speed of the wind. The objective of this project is to build a vehicle capable of travelling at a ground speed equal or superior to 100\% the speed of the wind, downwind.
\\ \hline
\textbf{Cost}: budget limit is \pounds850              
& 
The overall budget of the project is \pounds850, which gave the project a constraint on the amount that can feasibly be spent on the vehicle. For this criterion it was critical to complete a vehicle under the amount of money available.
\\ \hline
\textbf{Scale}: 80 cm propeller diameter       
& 
A main objective of the project is to test the vehicle in the R.J. Mitchell, hence a reasonably scaled model should be considered. Hence the propeller diameter was constrained to a maximum of 80 cm. The vehicle was thus built around this dimension. 
\\ \hline
\textbf{Durability}        
& 
The durability of the vehicle was important to the project as developing a vehicle that is resilient and does not deteriorate over time would allow for consistent testing. Having a prototype able to sustain wind tunnel testing was critical for this project. This way, data could be analysed and the aim of understanding the physical attributes of such vehicle, achieved. 
\\ \hline
\textbf{Safety}            
& 
Ensuring that the students and facility members were safe was a key focus for the project. This was achieved through thorough reviewing of the structural integrity of the prototype and rigorous compliance with relevant health and safety guidelines.
\\ \hline
\textbf{Modularity}        
& 
Having a vehicle that could be adapted easily helped the design process and allowed for updates to be made to the vehicle. Adapting the prototype during the project facilitated progress and enhanced project outcomes.
\\ \hline
\end{tabular}
\end{table}