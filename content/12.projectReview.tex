\subsection{Modularity}

Modularity was a key priority for the project, helping the progress in design and manufacture as the components could be easily adapted and replaced for performance upgrades. On many occasions, the vehicle needed re-designing due to issues with the structure, drivetrain or propeller. The modularity of the vehicle improved the response to the issues during the project as solutions could be easily implemented. Wind tunnel testing was also aided by the ability to connect and disconnect mechanisms of the vehicle with ease to obtain results for individual parts.

\subsection{3D printed propeller}

Initially, 3D printing the propeller blades was advised against, as the structural performance of the propeller blade would have been unknown before testing. Sufficient analytical results on its strength proved its viability. Following the design process, the blades were completed, significantly contributing to a low cost and weight of the vehicle. The propeller performed reliably in the wind tunnel and demonstrated clear signs of thrust, which was a positive output for the project.

\subsection{Performance}

The main criticism of the outcomes of the project was the vehicle performance as the maximum speed achieved was lower than the ground speed of the wind tunnel, as highlighted by a negative net thrust. The manufactured vehicle had many issues which were outlined by the testing data. Although the wind speed was not reached, invaluable information was gained about the functionality of the vehicle, understanding its limitations and where to improve.

\subsection{Drivetrain}

The drivetrain of the vehicle was unpredictable when running in the wind tunnel, with the chain falling off the sprockets at higher speeds. This restricted the data acquisition for the vehicle performance meaning a full range of results could not be collected, reducing the correlation between the data, especially at higher speeds.

\subsection{Propeller dimensions}
 
Another limitation for the vehicle was the diameter of the propeller in comparison to the vehicle dimensions. The reduced thrust from the propeller limited the performance of the vehicle.

\subsection{Innovation}

The downwind faster than the wind vehicles demonstrate significant potential for future engineering outputs, especially when relating drivetrain designs and ship propulsion systems to cleaner automated transport applications. The direct application of attaching a propeller to a vehicle would pose safety concerns. Nevertheless, inputting this technology into the sub systems of different machines could enhance operations and assist the movement to a more sustainable world.

The wind tunnel test demonstrated new applications of the theory of relative motion, testing the vehicle in cases that a representative of the real life applications of the vehicle. It was assumed that the vehicles of the Racing Aeolus competition have tested the vehicles internally before the competition days. This is the first documented wind tunnel facility test for a downwind faster than the wind vehicle at this scale. The two-wheel arm mount system in the wind tunnel proved a great help in testing on the rolling road, giving significant insight into improvements that could be applied to the vehicles, in a safe and controlled environment.

As highlighted throughout this report, the propeller was in part PLA 3D printed. This was, to the knowledge of the team, the first instance of 3D printed blades being held together with spars. The propeller could also be fully assembled or disassembled in minutes using two allen keys and a pair of pliers.

\subsection{Process}

When tasked with understanding the vehicle, many different inputs had to be considered, focusing on defining the basic principle of the downwind theory. The team organisation proved well thought out, with each student contributing significantly. Within the group communication between the drivetrain and the propeller team could have been improved to facilitate development. The process of the wind tunnel tests allowed for data to be acquired and with adaptation being able to be made between tests.  However, the vehicle could not be tested up to significant speeds due to structural limitations. 
Better performing outputs could have been achieved by building and testing an early small scale prototype. This would have allowed for better understanding of the practical aspects that come into play when designing a DWFTTW prototype. Earlier testing of the vehicle would have also been of great help had manufacturing been completed earlier.

\subsection{Sustainability}

To reduce the environmental impact of the project, it was important to think carefully about the approach to the design, and building of the vehicle. In itself, the concept of the DWFTTW vehicle is an environmentally friendly concept that could help advance sustainable technology in travel and energy production. Throughout this project, steps have been taken to reduce the environmental footprint.

One novelty of the design proposed by this project is the 3D printed Polylactic Acid (PLA) propeller. PLA 3D printing is a type of additive manufacturing and allows for much greater material savings than traditional methods like subtractive manufacturing when building a propeller out of wood or metal \cite{useOfBioPLA}. In a study on the environmental impact of 3D printing technologies, it was estimated that in 2025, for a good case scenario, savings of about 5\% in energy and CO2 emissions in the manufacturing industry worldwide can result from the use of additive manufacturing for creating new parts \cite{3DSustain}. However, PLA is indeed a biodegradable resource, but it takes an unusual amount of time to decompose. Efforts to develop better material that offer more biodegradability as well as ease of use in 3D printing, and low energy cost production. In a future iteration of this project, it would be interesting to study the possibility of using more environmentally friendly 3D printing filament to have a reduced carbon footprint.

The structure of the vehicle, as well as the hub and all main components are made from aluminium. Along with being the second most used metal in the world after steel, it is one of the most recyclable resource. Nearly 75\% of the all the aluminium produced in the last 100 years is still being used today. Aluminium can be recycled very easily, and endlessly, it would not lose its properties over iterations. This was one of the drivers when choosing aluminium for the main components of the vehicle.

During the year, the team has also focused on the small steps to ensure sustainable practice. Bulk orders were made for the parts that were required, to limit travel time. Additionally, the modularity of the vehicle means it can be taken apart easily, some parts reused in a future iterations, others recycled.

\subsection{Project Limitations and Opportunities for Further Work}
\begin{itemize}
    \item There were limitations with the planning of the project, and reliant on external sources to provide information and components for the vehicle that limited the progress of the project.
    \item More justification could have been put on drivetrain decisions, focusing more on numerical and simulation data than ease of manufacturing and adaptation
    \item Higher end components should have been acquired. This was mainly highlighted by the non-negligible portion of the budget remaining. 
\end{itemize}

\subsection{Budget}

Throughout the project, the budget was updated in a joint document which totalled the whole contributions. This was useful at the start of the project as budget could be assessed continually, especially when making important design decisions. However, minor issues arose concerning the budget following the submissions of manufacturing parts as the cost was unclear, meaning purchases could not be made not knowing the remainder of the budget. This held the project back as better quality bearings and wheels could have been bought, which would have improved the performance of the vehicle as the oscillations seen in the wind tunnel could have been minimised.

Whilst the overall budget was not known, the focus was on finding the cheapest option for the components to ensure that the budget was not exceeded. The conservative and optimistic options for the vehicle were continually monitored, so the options could be compared directly. This ensured that design and purchasing choices could be made based on estimates. Using this system, the focus could be primarily on one aspect over others.

\newpage
\section{Future work}

This section suggests improvements should more time have been available, or if the project was conducted again.

The most notable areas for improvement were the drivetrain and propeller as both sections were limited in performance when testing the vehicle. Swapping the bearings for more efficient ones would be highly beneficial. The bearings of the propeller shaft were found to be quite inefficient through manual turning, therefore, the priority would be to change then. The chain efficiency could also be improved by applying a chain guard to the vehicle, ensuring that all testing speeds could be tested. 

For the propeller, two main improvements would be possible. First, a larger propeller diameter would allow the vehicle to generate thrust more efficiently. This is in part due to the increased Reynolds number the propeller would be operating at. Also, as shown in \cite{drela20dead}, reducing the thrust coefficient, which can be done by increasing the propeller diameter, would yield far better results. 

%\subsection{Drivetrain}

Drivetrain inefficiencies played a significant roll in the negative net thrust generated by the vehicle. Improving the bearings and chain tensioning system would resolve much of these issues. In addition, one of the main problems were the wheels not being concentric enough which caused vibrations through the vehicle that transferred to the drivetrain. The use of better-quality tyres, ensuring them to run smoothly with a high precision of concentricity would be a good solution. Alternatively, manufacturing wheels from laser cut plywood and fitting a rubber outer would ensure a high degree of precision. This would remove the source of the vibrations which was the reason for the drivetrain reliability problems experienced in the wind tunnel.

Secondly, due to the lack of ability to change the sprockets during testing (one of the main reasons for choosing this method), the chain drive could be changed to a pure shaft drive, running vertically in the centre of the A-frame, with a 2nd gearbox at the top of the pylon connecting the prop shaft. This would be a more reliable drivetrain solution as any vibrations would not impact the shaft drive, as it did the chain drive (large lateral vibrations were experienced causing the chain to often derail and bearings to come loose). This would mean the vehicle could be tested at higher speeds without stability concerns.

With enhanced vehicle performance, more experimental tests could be completed in the wind tunnel, allowing for more comparisons to be made against predictions. Other potential forms of analysis of the vehicle could include the quantification of propeller shaft torque through experimental methods. This would allow validation of the underlying physics of the vehicle which have been detailed in this report.


