This group design project has focused on building a downwind faster than the wind (DWFTTW) vehicle. Although the working principles of this type of vehicle are not obvious, this concept has been proven to work on numerous occasions, both theoretically and experimentally. This has been done by multiple reliable sources. One of which is Prof Mark Drela's theoretical analysis of the working principle of the vehicle. Furthermore, a racing competition for this kind of vehicle has been held annually since 2018, in which multiple universities have taken part. The vehicle is composed of a set of wheels connected mechanically to a propeller. As the wind pushes the vehicle and speed builds up, the propeller receives torque from the wheels and propels the vehicle further. This allows the vehicle to reach speeds that can be faster than that of the wind. The theoretical aspect of this will be detailed in the Theory section.