% As a whole the project that was completed demonstrated many engineering decisions and processes allowing a vehicle to be manufactured and tested was a success as we completed the main goal of our project to test a DWFTTW vehicle in the wind tunnel. The limitations of the vehicle don’t completely encompass the success of the project as the understanding that we gained from our analysis would allow us to build a better vehicle given more time and budget. When purely basing the vehicle on performance, the vehicle didn’t complete the outlined goals at the start of the project. Nevertheless, the vehicles downfalls lied in many aspects, the chain system applied on the vehicle was not solid enough and varied performance significantly dependant on the tension. For the wind tunnel testing a limitation was also that there was no definitive way of measuring the effects of tension on the vehicle, this could have been completed by inputting a measurement system onto the chain tensioner system. This would have given the vehicle another input and more ability to understand the vehicle. The propeller was too small to generate the thrust needed, but this allowed us to gain an understanding. As a whole the group worked well with each other and completed a project and analysis that will help individuals and teams that are looking at building a vehicle of this kind.

% -Vehicle worked in wind tunnel
% -Visible signs of thrust
% -Overall net drag
% -Need to trust theory more
% -More time and budget would lead to second iteration

Over the duration of this project, a DWFTTW prototype was designed, manufactured, and tested to assess its performance. The design-phase utilised theory to inform on decision making, employing the use of numerous analytical and computational analyses to progress design and determine manufacturing methods. Experimental testing of the prototype was conducted in the RJ Mitchell wind tunnel at the University of Southampton to aid development of the structure and perform final testing. At the time this report was published, the experimental tests were believed by the authors to be the first instance of such vehicle being tested in a wind tunnel. Following tests, disparities were observed between the theoretical model and physical performance, primarily caused by unforeseen inefficiencies in the system. These were quantified in the form of an overall efficiency evaluated to equal 20\%. Although these issues were not resolved in time, analysis of the data generated areas of necessary improvement, yielding huge potential for future work.

In an effort to respect the budget and having no adequate alternatives, the propeller was designed and built in an innovative way, with PLA 3D printed blades. This provided a cost-effective, mass efficient and sustainable solution for custom geometries. Through experimental testing, the propeller was shown to be suitable for operational angular speeds up to 470 RPM, whilst maintaining a substantial margin in consideration of analytical performance predictions.

Inability to produce a positive net thrust on the vehicle in testing was attributed primarily to the larger than expected resistance observed in the drive train system. In addition, thrust was not generated efficiently due to the low Reynold's number the propeller was operating at. This was caused by using a small propeller to vehicle weight ratio but also a low operating RPM which was chosen as a compromise between safety and efficiency. These issues were only partially predicted by the design processes. As a result, the vehicle was calculated to only reach 56\% of the wind speed in 8 $\mathrm{m/s}$ wind.


